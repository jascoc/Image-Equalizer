\documentclass[a4paper, 12pt]{report}
\pdfpagewidth\paperwidth
\pdfpageheight\paperheight
\usepackage[italian]{babel}
\usepackage[utf8]{inputenc}
\usepackage[T1]{fontenc}
\usepackage{graphicx}
\usepackage{hyperref}
\usepackage{listings}
\usepackage{color}

\definecolor{dkgreen}{rgb}{0,0.6,0}
\definecolor{gray}{rgb}{0.5,0.5,0.5}
\definecolor{mauve}{rgb}{0.58,0,0.82}

\lstset{frame=tb,
  language=Vhdl,
  aboveskip=3mm,
  belowskip=3mm,
  showstringspaces=false,
  columns=flexible,
  basicstyle={\small\ttfamily},
  numbers=none,
  numberstyle=\tiny\color{gray},
  keywordstyle=\color{blue},
  commentstyle=\color{dkgreen},
  stringstyle=\color{mauve},
  breaklines=true,
  breakatwhitespace=true,
  tabsize=3
}
\begin{document}
	\title{Progetto finale di Reti Logiche - Equalizzatore dell'istogramma di una immagine}\author{Jasco Chen(10662235) Simone Coluccio(10633129)}\date{Anno 2020/2021}
	\maketitle

	\tableofcontents
	\chapter{Introduzione}
		\section{Descrizione generale}
			Il progetto si basa sull'implementazione, in VHDL con FPGA target :\\xc7a200tfbg484-1 e clock impostato a 100ns,
			di un metodo di equalizzazione dell'istogramma di una immagine.
			il metodo consiste nel ricalibrare il contrasto di una data immagine quando l'intervallo dei valori di intensità
			sono molto vicini effettuandone una distribuzione su tutto
			l'intervallo di intensità, al fine di incrementare il contrasto.
		\section{Algoritmo}
			l'algoritmo utilizzato è una versione semplificata della versione standard,\\
			è applicato su una scala di grigi a 256 livelli con intervallo tra [0 255],\\
		 	la trasformazione dei pixel avviene nel modo seguente:\\\\\\\par
					 DELTA VALUE = MAX PIXEL VALUE - MIN PIXEL VALUE\par
					 SHIFT LEVEL = (8 – FLOOR(LOG2(DELTA VALUE +1)))\par 
					 TEMP PIXEL = (CURRENT - MIN PIXEL VALUE) << SHIFT LEVEL\par
					 NEW PIXEL VALUE = MIN( 255 , TEMP PIXEL) \\\par
			Dove \textbf{MAX PIXEL VALUE} e \textbf{MIN PIXEL VALUE}, sono il massimo\par 
			e minimo valore dei pixel dell’immagine,\par
			\textbf{CURRENT PIXEL VALUE} è il valore del pixel da trasformare, e\par
			\textbf{NEW PIXEL VALUE} è il valore del nuovo pixel.\\\\\\\\
			%Esempio prima e dopo:\\\\
			%\includegraphics[width=208px, height=117px]{before} \includegraphics[width=208px, height=117px]{after}
		\section{Interfaccia del componente}
			Il componente descritto ha la seguente interfaccia:\\\\
				entity project\_reti\_logiche is\par
				port (\par
					i\_clk : in std\_logic;\par
					i\_rst : in std\_logic;\par
					i\_start : in std\_logic;\par
					i\_data : in std\_logic\_vector(7 downto 0);\par
					o\_address : out std\_logic\_vector(15 downto 0);\par
					o\_done : out std\_logic;\par
					o\_en : out std\_logic;\par
					o\_we : out std\_logic;\par
					o\_data : out std\_logic\_vector (7 downto 0)\\
				);\\
				end project\_reti\_logiche;\\\\
			in particolare:\\\par
				• i\_clk è il segnale di CLOCK in ingresso generato dal TestBench;\par
				• i\_rst è il segnale di RESET che inizializza la macchina pronta\par\par per ricevere il primo segnale di START;\par
				• i\_start è il segnale di START generato dal Test Bench;\par
				• i\_data è il segnale (vettore) che arriva dalla memoria in seguito ad una\par\par richiesta di lettura;\par
				• o\_address è il segnale (vettore) di uscita che manda l’indirizzo alla\par\par memoria;\par
				• o\_done è il segnale di uscita che comunica la fine dell’elaborazione e il\par\par dato di uscita scritto in memoria;\par
				• o\_en è il segnale di ENABLE da dover mandare alla memoria per poter\par\par comunicare (sia in lettura che in scrittura);\par
				• o\_we è il segnale di WRITE ENABLE da dover mandare alla memoria\par\par (=1) per poter scriverci. Per leggere da memoria esso deve essere 0;\par
				• o\_data è il segnale (vettore) di uscita dal componente verso la memoria.\\
		\section{Dati e descrizione della Memoria}
			Il modulo implementato legge l’immagine da una memoria in cui è memorizzata,
			sequenzialmente e riga per riga, l’immagine da elaborare. Ogni byte corrisponde ad un pixel
			dell’immagine.\\
			La dimensione della immagine è definita da \textbf{2 byte}, memorizzati a partire dall’\textbf{indirizzo 0}.\\ Il
			byte all’indirizzo 0 si riferisce alla dimensione di colonna; il byte nell’indirizzo 1 si riferisce
			alla dimensione di riga.\\ La \textbf{dimensione massima} dell’immagine è \textbf{128x128 pixel}.\\
			L’immagine è memorizzata a partire dall’indirizzo 2 e in byte contigui. Quindi il byte
			all’indirizzo 2 è il primo pixel della prima riga dell’immagine.\\\\
			I pixel dell'immagine, ciascuna di \textbf{8 bit}, sono memorizzari in unasono memorizzati in una
			memoria con indirizzamento al Byte partendo dalla \textbf{posizione 0}: il byte in posizione 0 si
			riferisce al numero di \textbf{colonne} (N-COL), il byte in posizione 1 si riferisce al numero di \textbf{righe}
			(N-RIG).\\
			I pixel del’immagine, ciascuno di un \textbf{8 bit}, sono memorizzati in memoria con indirizzamento
			al Byte partendo dalla posizione \textbf{2}.\\
			I pixel della immagine equalizzata, ciascuno di un \textbf{8 bit}, sono memorizzati in memoria con
			indirizzamento al Byte partendo dalla \textbf{posizione}: 2+(N-COL*N-RIG)+1.\\\\\\
			\textbf{ESEMPIO} di allocazione nella RAM di una immagine \textbf{2x2}:\\\par
				\begin{center}
				\begin{tabular}{|c | c|l}
					 \hline Ind. Mem. & Valore \\ \hline 
								0 & 2\\ \hline
								1 & 2\\ \hline
								2 & 46\\ \hline
								3 & 131\\ \hline
								4 &	62 \\ \hline
								5 & 89\\ \hline
								6 & 0\\ \hline
								7 & 255\\ \hline
								8 & 62 \\ \hline
								9 & 172\\ \hline
				\end{tabular}
				\end{center}
				dove le 
				\textbf{dimensioni} della immagine è definita negli indirizzi \textbf{0} e \textbf{1}, i \textbf{pixel} dell'immagine arrivati tramite in input dall'indirizzo \textbf{[2 5]} 
				e i pixel equalizzati dall'indirizzo \textbf{[6 9]} 
	\chapter{Design}
		\section{Segnali implementati}
			Per il memorizzare i valori dei pixel abbiamo scelto di utilizzare registri \textbf{principali} e registri \textbf{ausiliari}, quest'ultimi forniscono una sorta di supporto per il corretto funzionamento della macchina a 					stati.\\
			I \textbf{registri} utilizzati per il calcolo diretto dell'algoritmo hanno nomi analoghi alle variabili dell'algoritmo mentre quelle utilizzati in modo ausiliare hanno un indice numerico.\\
			I segnali di \textbf{load} sono utilizzati per il caricamento di un valore nel registro con il nome simile a quello del segnale.\\
			Segnali \textbf{max} e \textbf{min} sono segnali per notificare la macchina se il valore entrante è un massimo oppure un minimo assoluto.\\
			Segnali \textbf{sel} insieme ai \textbf{mux} sono utilizzati per gestire gli input dei \textbf{multiplexer}.\\
			Segnali \textbf{sum} e \textbf{sub} per gestire rispettivamente la somma e sottrazione.\\ 
			\subsection{Scelte progettuali}
				Per la moltiplicazione abbiamo scelto di procedere con un algoritmo, che dati due input \textbf{n} e \textbf{m} sommava \textbf{m}-volte \textbf{n} a se stesso.\\
				I cicli li abbiamo gestiti tramite l'utilizzo di un \textbf{multiplexer} insieme ad un \textbf{sommatore} oppure \textbf{sottrattore} che decrementava oppure incrementava l'output del multiplexer e faceva un controllo:\\
				E infine per il \textbf{FLOOR(LOG2)} dato che i risultato potevano variare solo in un intervallo finito e relativamente piccolo \textbf{[0 8]} abbiamo implementato dei controlli a soglia:\\
					\begin{lstlisting}
						process(i_clk, i_rst)
						    begin
						        if(i_rst = '1') then
						            floor_log <= "00000000";
						        elsif i_clk'event and i_clk = '1' then
						            if(r10_load = '1') then
						                if   (delta_value1 < "0000000000000010") then
						                    floor_log <= "00000000";
						                elsif(delta_value1 < "0000000000000100") then
						                    floor_log <= "00000001";
						                elsif(delta_value1 < "0000000000001000") then
						                    floor_log <= "00000010";
						                elsif(delta_value1 < "0000000000010000") then
						                    floor_log <= "00000011";
						                elsif(delta_value1 < "0000000000100000") then
						                    floor_log <= "00000100";
						                elsif(delta_value1 < "0000000001000000") then
						                    floor_log <= "00000101";
						                elsif(delta_value1 < "0000000010000000") then
						                    floor_log <= "00000110";
						                elsif(delta_value1 < "0000000100000000") then
						                    floor_log <= "00000111";
						                elsif(delta_value1 = "0000000100000000") then
						                    floor_log <= "00001000";
						                end if;
						            end if;
						        end if;
						    end process;    
					\end{lstlisting}
		\section{Stati della Macchina}
			Abbiamo progettato una macchina a stati di basso livello, spezzando ogni operazione in uno stato a parte per una maggiore chiarezza:\\
			
			\begin{itemize}
 				\item \textbf{START}: stato iniziale dove attendiamo un segnale di i\_start = 1;
				\item \textbf{S1}: leggiamo il valore all'indirizzo di memoria = 0;
                \item \textbf{S2}: carichiamo il primo valore letto nel registro \textbf{col}, poi carichiamo 0 nel registro \textbf{img\_dim} e in fine leggiamo il valore all'indirizzo di memoria = 1;
				\item \textbf{S3}: carichiamo il valore letto nello stato precedente nel registro \textbf{row};
				\item \textbf{S4}: cominciamo a decrementare \textbf{row};
				\item \textbf{S5}: cominciamo il ciclo di decremento di \textbf{row} e incremento del suo \textbf{col} del suo stesso valore per fare la moltiplicazione;
				\item \textbf{S6}: il risultato della moltiplicazione viene salvato nel registro \textbf{o\_r5};
				\item \textbf{S7}: comincia il ciclo per la lettura di ogni singolo pixel dell'immagine;
				\item \textbf{S8}: si salva il valore appena letto in \textbf{o\_r2} e decremento il contatore della dimensione dell'immagine
				\item \textbf{S9}: assegnamo il primo pixel ai registri \textbf{max} e \textbf{min};
				\item \textbf{S10}: incrementiamo l'indirizzo per leggere il valore successivo;
				\item \textbf{S11}: assegniamo a o\_address l'indirizzo incrementato e leggiamo il valore;
				\item \textbf{S12}: si salva il nuovo pixel in \textbf{o\_r2};
				\item \textbf{S13}: attendiamo il caricamento del pixel per una corretta valutazione del max e min;
				\item \textbf{S14}: stato in cui aggiorniamo \textbf{min} nel caso in cui non avessimo letto tutti i pixel si ritorna allo stato \textbf{S10};
				\item \textbf{S15}: stato in cui aggiorniamo \textbf{max} nel caso in cui non avessimo letto tutti i pixel si ritorna allo stato \textbf{S10};
				\item \textbf{S16}: calcoliamo il \textbf{delta\_value};
				\item \textbf{S17}: calcoliamo il \textbf{delta\_value + 1} e si imposta l'indirizzo corretto di \textbf{scrittura};
				\item \textbf{S18}: calcoliamo il \textbf{log(delta\_value + 1)} e si imposta l'indirizzo correto di \textbf{lettura};
				\item \textbf{S19}: calcoliamo lo \textbf{shift\_lvl} e carichiamo \textbf{o\_r5} in \textbf{o\_r15}; 
				\item \textbf{S20}: cominciamo il ciclo di \textbf{scrittura} e \textbf{lettura};
				\item \textbf{S21}: assegniamo l'indirizzo di \textbf{lettura} a \textbf{o\_address};
				\item \textbf{S22}: effettuiamo la richiesta di \textbf{lettura};
				\item \textbf{S23}: salviamo il valore letto in \textbf{o\_r16};
				\item \textbf{S24}: calcoliamo il valore del pixel corrente meno il \textbf{min};
				\item \textbf{S25}:	shiftiamo il valore ottenuto e assegniamo a \textbf{o\_address} l'indirizzo di \textbf{scrittura};
				\item \textbf{S26}: incrementiamo il contatore del ciclo e carichiamo il valore ottenuto dall'algoritmo in \textbf{o\_data} per la \textbf{scrittura};
				\item \textbf{S27}: effetuiamo la scrittura in memoria;
				\item \textbf{S28}: attendiamo l'incremento del contatore per una corretta valutazione del segnal di fine \textbf{scrittura};
				\item \textbf{FINISH}: settiamo \textbf{o\_done} = 1 e ritorniamo allo stato \textbf{START}.
			\end{itemize}
			\begin{center}
				\includegraphics[width=400px, height=168px]{macchina}
			\end{center}
	\chapter{Testing}
		\section{Test Base}
			Abbiamo usato i test base forniti dalla consegna come primo step per verificare il corretto funzionamento del progetto.\\
			Sotto è riportato il risultato del \textbf{Behavioural} con input presi di una immagine \textbf{4x3}:\\
			\includegraphics[width=530px, height=300px]{TestBase}\\\\
			\textbf{valori utilizzati:}\\
			\begin{center}
			\begin{tabular}{|c | c|}
					 \hline Ind. Mem. & Valore \\ \hline 
								0 & 4\\ \hline
								1 & 3\\ \hline
								2 & 0\\ \hline
								3 & 0\\ \hline
								4 &	0 \\ \hline
								5 & 0\\ \hline
								6 & 128\\ \hline
								7 &  128\\ \hline
								8 &  128 \\ \hline
								9 &  128\\ \hline
								10 & 255\\ \hline
								11 & 255\\ \hline
								12 & 255 \\ \hline
								13 & 255\\ \hline
								14 & 0\\ \hline
								15 & 0\\ \hline
								16 & 0 \\ \hline
								17 & 0\\ \hline
								18 & 128\\ \hline
								19 & 128\\ \hline
								20 &	128 \\ \hline
								21 & 128\\ \hline
								22 & 255\\ \hline
								23 &	255 \\ \hline
								24 & 255\\ \hline
								25 & 255\\ \hline
				\end{tabular}
				\end{center}
		\section{Test casi limite}
			Abbiamo identificato due pricipali casi limite:\\\par
			\begin{enumerate}
 				\item caso immagine con \textbf{dimensione 0}, ovvero se uno dei valori presenti nei primi due indirizzi di memoria, che corrispondo alla dimensione, fosse 0;
					  
 				\item caso immagine con \textbf{dimensione 1}, ovvero i valori presenti nei primi due indirizzi di memoria, che corrispondono alla sua dimensione, fossero entrambi 1.
			\end{enumerate}
			\subsection{Immagine dimensione 0xDC}
				In questo caso, dato che nel nostro \textbf{design} della macchina a stati abbiamo gestito tale caso nello stato \textbf{S7} dove il risultato della motiplicazione risulta \textbf{0} e porta la macchina
				direttamente allo stato \textbf{FINISH} con \textbf{o\_we} sempre basso, il test risulta corretto.\\
				Sotto è riportato il risultato del \textbf{Behavioural} con input presi di una immagine \textbf{2x0}:\\\\
				\includegraphics[width=530px, height=300px]{img2x0}
					\textbf{valori utilizzati:}\\
						\begin{center}
						\begin{tabular}{|c | c|}
					 		\hline Ind. Mem. & Valore \\ \hline 
								0 & 2\\ \hline
								1 & 0\\ \hline
						\end{tabular}
						\end{center}	
			\subsection{Immagine dimensione 1x1}
				Trattato come caso limite ma non tanto da dover scalare il design, è dunque possibile processarlo come una qualsiasi immagine di dimensioni maggiori di 0.
				Sotto è riportato il risultato del \textbf{Behavioural} con input presi di una immagine \textbf{1x1}:\\\\
					\includegraphics[width=530px, height=300px]{img1x1}
					\textbf{valori utilizzati:}\\
						\begin{center}
						\begin{tabular}{|c | c|}
					 		\hline Ind. Mem. & Valore \\ \hline 
								0 & 1\\ \hline
								1 & 1\\ \hline
								2 & 31\\ \hline
								3 & 0\\ \hline
						\end{tabular}
						\end{center}
		\section{Test autogenerati da script python}
			Abbiamo in fine usuffruito di un generatore di casi di test(\href{https://github.com/davidemerli/RL-generator-2020-2021}{git repository}) a singole e multiple immagini con esisito positivo. 
	\chapter{Conclusione}
			Il componente non funziona in \textbf{post sinstesi} dato che ci siamo accorti della presenza di \textbf{latch} quando assegniamo un valore direttamente negli
			stati :
				durante la descrizione del \textbf{current state} ai registri che manipolano gli idirizzi (3 \textbf{latch} ognuno di 16: uno per il registro \textbf{address}, uno per \textbf{out\_address} e uno per \textbf{o\_address}).\\\\\\
			Il problema è risolvibile tramite l'assegnameto dei valori tramite \textbf{segnali} che verranno gestiti nel \textbf{datapath}(tramite i \textbf{process}) evitando 
			l'assegnamento diretto nella descrizione del \textbf{current state}.\\ Abbiamo implementato una bozza della soluzione sopracitata ed effettivamente il numero di \textbf{latch} scompare e il componente si sintetizza
			correttamente. Tuttavia tale implementazione fa variare il risultato dell'algoritmo e lo porta a dei valori non corretti, abbiamo dunque capito che fosse necessario una riprogettazione abbastanza sostanziosa del
			componente. Per problemi di tempistica dunque abbiamo optato di lasciare il componente funzionante solo in \textbf{pre sintesi} e riprogettarlo in futuro.
\end{document}